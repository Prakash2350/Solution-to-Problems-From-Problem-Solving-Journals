\documentclass[12pt]{article}
\usepackage{amsmath}
\begin{document}

\textit{Proposed Solution to \#5761 SSMJ }\\ \\
\textit{Solution proposed by Prakash Pant, Mathematics Initiatives in Nepal, Bardiya, Nepal.}\\ \\
\textit{Problem proposed by Narendra Bhandari and Yogesh Joshi, Nepal.
} \\ \\
\textbf{Statement of the Problem:}
\[ \sum_{n=0}^{\infty} \binom{2n}{n} \frac{\left( H_{[\frac{n}{2}]} - H_{ [ \frac{n-1}{2}]} \right)} {4^n (6n+3)} + \int_0^{\frac{\pi}{4}} \frac{4 y \sec(y) dy}{\sqrt{ 9 \cos(2y)}} = \zeta{(2)}\] 
\[ \text{where } H_{[n]} = \int_0^1 \frac{1-x^n}{1-x} \text{dx and } \zeta(n) = \sum_{k=1}^{\infty} \frac{1}{k^n} \text{ is Reimann zeta function for n} > 1 .\]

\textbf{Solution of the Problem:}
Name sum as S and integral as I such that our answer is S + I. \\ 
\[S =  \sum_{n=0}^{\infty} \binom{2n}{n} \frac{\left( H_{[\frac{n}{2}]} - H_{ [ \frac{n-1}{2}]} \right)} {4^n (6n+3)} \]
\[ I =  \int_0^{\frac{\pi}{4}} \frac{4 y \sec(y) dy}{\sqrt{ 9 \cos(2y)}} \]  
Let us find try to find I. \\
\[ \text{ Using} \cos(2y)=\cos^2(y) - \sin^2(y)\text{, we get } \]


\[ I=  \int_0^{\frac{\pi}{4}} \frac{4 y \sec(y) dy}{3 \sqrt{ \cos^2(y) - \sin^2(y) }} \]  

\[ \text{ Dividing numerator and denominator by } \cos(y) \]
 
\[ I= \frac{4}{3} \int_0^{\frac{\pi}{4}}  \frac{ y \sec^2(y) dy}{ \sqrt{ 1 - \tan^2(y) }} \]  

 Making an u-substitution as u = $\tan(y) => du = \sec^2(y)$ dy
 . And the integral goes from 0 to 1. 
 
 \[ I= \frac{4}{3} \int_0^{1} \frac{ \arctan(u) du}{ \sqrt{ 1 - u^2}} \]    
Using Integration by Parts,
  \[ I= \frac{4}{3} \left(  \arctan(u) \arcsin(u) \Big|_0^1 -  \int_0^{1} \frac{\arcsin(u)}{1+u^2} du \right) \] 
 \[ I = \frac{4}{3} \times \frac{\pi^2}{8}  - \frac{4}{3} \int_0^{1} \frac{\arcsin(u)}{1+u^2} du \]
 \[ I = \frac{\pi^2}{6}  - \frac{4}{3} \int_0^{1} \frac{\arcsin(u)}{1+u^2} du \] 
Now, since $ \zeta(2) = \sum_{k=1}^{\infty} \frac{1}{k^2} = \frac{\pi^2}{6}$, the problem can be written as: 
 \[ \sum_{n=0}^{\infty} \binom{2n}{n} \frac{\left( H_{[\frac{n}{2}]} - H_{ [ \frac{n-1}{2}]} \right)} {4^n (6n+3)} + \frac{\pi^2}{6}  - \frac{4}{3} \int_0^{1} \frac{\arcsin(x)}{1+x^2} dx = \frac{\pi^2}{6}\]  
which can further be modified as 
  \[ \sum_{n=0}^{\infty} \binom{2n}{n} \frac{\left( H_{[\frac{n}{2}]} - H_{ [ \frac{n-1}{2}]} \right)} {4^n (6n+3)} = \frac{4}{3} \int_0^{1} \frac{\arcsin(x)}{1+x^2} dx\]  
 
\[ \text{ Now, we can use the taylor expansion of}  \arcsin(x) \text{ for } |x|<1:\]
\[  \arcsin(x) = \sum_{n=0}^{\infty} \frac{1}{4^n} \binom{2n}{n} \frac{x^{2n+1}}{2n+1} \]  
we get,
   \[ \sum_{n=0}^{\infty} \binom{2n}{n} \frac{\left( H_{[\frac{n}{2}]} - H_{ [ \frac{n-1}{2}]} \right)} {4^n (6n+3)} = \frac{4}{3} \int_0^{1} \left{ \frac{1}{1+x^2} \sum_{n=0}^{\infty}  \frac{1}{4^n} \binom{2n}{n} \frac{x^{2n+1}}{2n+1} \right} dx\]  
 Since the sum is in the world of n, we can take $ \frac{1}{1+x^2}$ inside the sum as a constant.
  \[ \sum_{n=0}^{\infty} \binom{2n}{n} \frac{\left( H_{[\frac{n}{2}]} - H_{ [ \frac{n-1}{2}]} \right)} {4^n (6n+3)} = \frac{4}{3} \int_0^{1}   \sum_{n=0}^{\infty}  \frac{1}{4^n} \binom{2n}{n} \frac{1}{1+x^2} \frac{x^{2n+1}}{2n+1}   dx\]  
  Interchanging sum and integral using dominated convergence theorem,
    \[ \sum_{n=0}^{\infty} \binom{2n}{n} \frac{\left( H_{[\frac{n}{2}]} - H_{ [ \frac{n-1}{2}]} \right)} {4^n (6n+3)} = \frac{4}{3}  \sum_{n=0}^{\infty}   \int_0^{1}  \frac{1}{4^n} \binom{2n}{n} \frac{1}{1+x^2} \frac{x^{2n+1}}{2n+1}   dx\]  
 Taking constants out of the integration
    \[ \sum_{n=0}^{\infty} \binom{2n}{n} \frac{\left( H_{[\frac{n}{2}]} - H_{ [ \frac{n-1}{2}]} \right)} {4^n (6n+3)} = 4  \sum_{n=0}^{\infty} \binom{2n}{n} \frac{1}{4^n (6n+3)}  \int_0^{1}  \frac{x^{2n+1}}{1+x^2}    dx\]  
Now, it's enough to prove that:
\[   4 \int_0^{1}  \frac{x^{2n+1}}{1+x^2}   =  H_{[\frac{n}{2}]} - H_{ [ \frac{n-1}{2}]} \]
 \[ \text{ Let } y=x^2 => y^{\frac{1}{2}} = x => \frac{1}{2}y^{-\frac{1}{2}} dy = dx  \]
 \[ = 2 \int_0^{1} \frac{y^{n+\frac{1}{2}}}{1+y} y^{-\frac{1}{2}} dy \]
 \[ = 2 \int_0^{1} \frac{y^{n}}{1+y}  dy \]
 Using the formula for geometric sum as $|y|<1$
\[ = 2\int_0^{1} y^n \sum_{r=0}^{\infty} (-y)^r  dy\]
Since the sum is in the world of r, we can take the constant $y^n$ inside the sum,
\[ = 2\int_0^{1} \sum_{r=0}^{\infty} (-1)^r y^{n+r} dy \]
Interchanging the sum and integral using dominated convergence theorem,
\[ = 2 \sum_{r=0}^{\infty} \int_0^{1} (-1)^r y^{n+r} dy \]
\[ = 2 \sum_{r=0}^{\infty}  (-1)^r \left( \frac{y^{n+r+1}}{n+r+1} \right) \Big|_0^1  \]

\[ = 2 \left( \sum_{r=0}^{\infty}  (-1)^r  \frac{1}{n+r+1} \right)   \] 


\[ = 2 \left( \sum_{r=0}^{\infty} \frac{1}{n+2r+1} - \frac{1}{n+2r+2}   \right) \] 
Changing the starting point of sum from r=0 to r=1 

\[ = 2 \left( \sum_{r=1}^{\infty} \frac{1}{n+2r-1} -  \frac{1}{n+2r}   \right) \] 

\[ =  \sum_{r=1}^{\infty} \frac{1}{r+\frac{n-1}{2}} -  \frac{1}{r+\frac{n}{2}}   \right) \] 

\[ \text{ Now, we know that } \int_0^1 \frac{1-t^x}{1-t} dt = H_x = \sum_{r=1}^{\infty} \frac{1}{r}-\frac{1}{r+x} \]


\[   \sum_{r=1}^{\infty} \frac{1}{r+\frac{n-1}{2}} -  \frac{1}{r+\frac{n}{2}}    =  \sum_{r=1}^{\infty} \left( \frac{1}{r} -  \frac{1}{r+\frac{n}{2}}   \right) - \left( \sum_{r=1}^{\infty} \frac{1}{r}-  \frac{1}{r+\frac{n-1}{2}} \right) \] 
\[ = H_{[\frac{n}{2}]} - H_{[ \frac{n-1}{2}]} \] which proves the problem. 


\end{document}