\documentclass[12pt]{article}
\usepackage{amsmath}
\begin{document}

\textit{Proposed Solution to \#U654 Undergraduate Problems, Mathematical Reflections 1 (2024) }\\ \\
\textit{Solution proposed by Prakash Pant, Mathematics Initiatives in Nepal, Bardiya, Nepal.}\\ \\
\textit{Problem Proposed by Vasile Mircea Popa, Sibiu, Romania.} \\ \\
\textbf{Statement of the Problem:}
\[ \text{ Evaluate} : \]
\[ I = \int_0^1 \frac{x \sqrt{x} \ln(x)}{x^2-x+1} dx\]

\textbf{Solution of the Problem:} \\
Multiply numerator and denominator by 1+x, 
\[ I = \int_0^1 \frac{(x^{\frac{3}{2}}+x^{\frac{5}{2}}) \ln(x)}{1+x^3} dx \]
Substitute $x^3$ as x ,
\[ I = \frac{1}{9} \int_0^1 \frac{(x^{\frac{1}{2}}+x^{\frac{5}{6}}) \ln(x)}{1+x} x^{\frac{-2}{3}} dx = \frac{1}{9} \int_0^1 \frac{(x^{\frac{-1}{6}}+x^{\frac{1}{6}}) \ln(x)}{1+x} dx \]
Since x is going from 0 to 1, using the geometric infinite series expansion for $\frac{1}{1+x}$,
\[ I = \frac{1}{9} \int_0^1 \left[ (x^{\frac{-1}{6}}+x^{\frac{1}{6}}) \ln(x) \sum_{n=0}^{\infty} (-x)^n \right] dx \]
Taking the constants inside the sum and then interchanging sum and integral using dominated convergence theorem, 
\[ I = \frac{1}{9} \sum_{n=0}^{\infty} (-1)^n  \int_0^1  (x^{n-\frac{1}{6}}+x^{n+\frac{1}{6}}) \ln(x)  dx \]
Using $ \int_0^1 \ln^n(x) x^m dx = (-1)^n \frac{\Gamma(n+1)}{(m+1)^{n+1}}$,
\[ I = \frac{1}{9} \sum_{n=0}^{\infty} (-1)^n  \left( \frac{(-1)^1 \Gamma(2)}{(n+\frac{5}{6})^2} + \frac{(-1)^1 \Gamma(2)}{(n+\frac{7}{6})^2} \right) = \frac{1}{9} \sum_{n=0}^{\infty} (-1)^{n+1}  \left( \frac{1}{(n+\frac{5}{6})^2} + \frac{1}{(n+\frac{7}{6})^2} \right)  \]
\[ I = \frac{1}{9} \left[  \sum_{n=0}^{\infty}  (-1) \left(  \frac{1}{(2n+\frac{5}{6})^2} + \frac{1}{(2n+\frac{7}{6})^2} \right) +  \sum_{n=0}^{\infty}  \left( \frac{1}{(2n+\frac{11}{6})^2} + \frac{1}{(2n+\frac{13}{6})^2} \right)  \right] \]
\[ I = \frac{1}{36} \left[  \sum_{n=0}^{\infty}  (-1) \left(  \frac{1}{(n+\frac{5}{12})^2} + \frac{1}{(n+\frac{7}{12})^2} \right) +  \sum_{n=0}^{\infty}  \left( \frac{1}{(n+\frac{11}{12})^2} + \frac{1}{(n+\frac{13}{12})^2} \right)  \right] \]
Using $\psi'(x)= \sum_{n=0}^{\infty} \frac{1}{(n+x)^2}$,
\[ I = \frac{1}{36} \left[   - \psi'\left( \frac{5}{12}\right) - \psi'\left(\frac{7}{12}\right)  +   \psi'\left(\frac{11}{12}\right) +  \psi'\left(\frac{13}{12} \right)  \right] \]
Using $\psi'(1+x)= \psi'(x) - \frac{1}{x^2}$ ,
\[ I = \frac{1}{36} \left[   - \psi'\left( \frac{5}{12}\right) - \psi'\left(\frac{7}{12}\right)  +   \psi'\left(\frac{11}{12}\right) +  \psi'\left(\frac{1}{12}\right) - 144   \right] \]
Using $ \psi'(x)+\psi'(1-x) = \pi^2 \text{ cosec}^2(\pi x) $
\[ I = \frac{1}{36} \left[   -\pi^2 \text{cosec}^2\left(\frac{5 \pi}{12}\right)  + \pi^2 \text{cosec}^2\left( \frac{\pi}{12} \right) - 144   \right] = \frac{\left[  \text{cosec}^2\left(\frac{ \pi}{12} \right)  - \text{cosec}^2\left(\frac{5 \pi}{12} \right) \right] \pi^2}{36} - 4 \]
We can calculate that $\left[  \text{cosec}^2\left(\frac{ \pi}{12} \right)  - \text{cosec}^2\left(\frac{5 \pi}{12} \right) \right] = 8 \sqrt{3} $, 
\[ I = \frac{8 \sqrt{3} \pi^2}{36} - 4 = \frac{2\pi^2}{3 \sqrt{3}} - 4 \] 
which is the solution to the given integral. 

\end{document}