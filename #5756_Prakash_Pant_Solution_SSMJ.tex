\documentclass[12pt]{article}
\usepackage{amsmath}
\begin{document}

\textit{Proposed Solution to \#5756 SSMJ }\\ \\
\textit{Solution proposed by Prakash Pant, Mathematics Initiatives in Nepal, Bardiya, Nepal.}\\ \\
\textit{Problem proposed by  Toyesh Prakash Sharma (Undergraduate Student) Agra College, India.} \\ \\
\textbf{Statement of the Problem:}
\[ \text{Calculate } T = \int_0^{\infty} \frac{dx}{x^2 (\tan^2(x)+\cot^2(x))} \] 
\textbf{Solution of the Problem:}
\[ \text{ Using} \tan(x)=\frac{\sin(x)}{\cos(x)} \text{ and} \cot(x)= \frac{\cos(x)}{\sin(x)} \text{, we get } \]

\[ = \int_0^{\infty} \frac{\sin^2(x) \cos^2(x)}{x^2} . \frac{dx}{(\sin^4(x)+\cos^4(x)} \]  
\[ \text{ Using} \sin(2x) = 2 \sin(x) \cos(x) \text{ and } a^2+b^2 = (a+b)^2-2ab  \text{, we get } \]
\[ = \int_0^{\infty} \frac{\sin^2(2x)}{(2x)^2} . \frac{dx}{(\sin^2(x)+\cos^2(x))^2 - 2 \sin^2(x) \cos^2(x) } \]
\[ \text{ Using} \sin^2(x)+\cos^2(x) = 1 \text{ and } \sin(2x) = 2 \sin(x) \cos(x), we get \]
\[ = \int_0^{\infty} \frac{\sin^2(2x)}{(2x)^2} . \frac{dx}{ \left( 1- \frac{1}{2} \sin^2(2x) \right)  } \]
 Now, we make a u-substitution such that u = 2x .  This implies $\frac{du}{2}= dx $ . The integral now goes from 0 to infinity. 
\[ = \int_0^{\infty} \frac{\sin^2(u)}{u^2} . \frac{du}{ \left( 2- \sin^2(u) \right)  } \]
\[ \text{ Using } \sin^2(x)+\cos^2(x) = 1, \text{ we get}  \]
\[ = \int_0^{\infty} \frac{\sin^2(u)}{u^2} . \frac{du}{ \left( 1+ \cos^2(u) \right)  } \]
 Lobachevsky's Formula states that if $0 \le u < \infty$ , f(u)=f(-u) and f(u+$\pi$ k ) = f(u), then 
 \[ \int_0^{\infty} \frac{\sin^2(u)}{u^2} f(u) du = \int_0^{\frac{\pi}{2}} f(u) du \]
\[ \text{Here, } f(u) = \frac{1}{1+ \cos^2(u)} \text{ satisfies the conditions of Lobachevsky's Formula. Therefore, } \]
\[ \int_0^{\infty} \frac{\sin^2(u)}{u^2} . \frac{du}{ \left( 1+ \cos^2(u) \right)  } = \int_0^{\frac{\pi}{2}} \frac{du}{1+\cos^2(u)} \]
Multiplying numerator and denominator by $ \sec^2(u) $, we get 
  \[  = \int_0^{\frac{\pi}{2}} \frac{ \sec^2(u) du}{\sec^2(u)+1} \]
Using $ \sec^2(u)-\tan^2(u) = 1 $, we get 
\[  = \int_0^{\frac{\pi}{2}} \frac{ \sec^2(u) du}{2+\tan^2(u)} \]		
Now, we make a y-substitution such that y = tan(u). This implies 	dy= $\sec^2(u)$ du . The integral now goes from 0 to infinity. 
\[  = \int_0^{\infty} \frac{ dy}{2+y^2} \]	
\[ \text{ We know, } \int \frac{1}{a^2+x^2} dx = \frac{1}{a} arctan \left( \frac{x}{a} \right)+ c \]
Thus,
\[  \int_0^{\infty} \frac{ dy}{2+y^2} = \frac{1}{\sqrt{2}} arctan\left( \frac{x}{\sqrt{2}} \right) \Big| _0^{\infty} \]	
\[ = \frac{1}{\sqrt{2}} \left( \frac{\pi}{2} - 0 \right) = \frac{\pi}{2 \sqrt{2}} \]
\[ \text{ Hence, } T =\int_0^{\infty} \frac{dx}{x^2 (\tan^2(x)+\cot^2(x))} =  \frac{\pi}{2 \sqrt{2}} \]
 



\end{document}